\documentclass[aspectratio=169]{beamer}
\usepackage{tabularx}
\usepackage{graphicx}
\usepackage{eso-pic}

\usepackage{minted}
\usepackage{hyperref}
\usepackage{animate}

\usepackage{tcolorbox}

\makeatletter
\newenvironment{myitemize}{%
   \setlength{\topsep}{0pt}
   \setlength{\partopsep}{0pt}
   \renewcommand*{\@listi}{\leftmargin\leftmargini \parsep\z@ \topsep\z@ \itemsep\z@}
   \let\@listI\@listi
   \itemize
}{\enditemize}
\makeatother

\graphicspath{{img/}}

\usetheme{Warsaw}
\usemintedstyle{monokai}

\setbeamercolor{normal text}{fg=white,bg=black!90}
\setbeamercolor{structure}{fg=white}
\setbeamercolor{alerted text}{fg=red!85!black}
\setbeamercolor{item projected}{use=item,fg=black,bg=item.fg!35}
\setbeamercolor*{palette primary}{use=structure,fg=structure.fg}
\setbeamercolor*{palette secondary}{use=structure,fg=structure.fg!95!black}
\setbeamercolor*{palette tertiary}{use=structure,fg=structure.fg!90!black}
\setbeamercolor*{palette quaternary}{use=structure,fg=structure.fg!95!black,bg=black!80}
\setbeamercolor*{framesubtitle}{fg=white}
\setbeamercolor*{block title}{parent=structure,bg=black!60}
\setbeamercolor*{block body}{fg=black,bg=black!10}
\setbeamercolor*{block title alerted}{parent=alerted text,bg=black!15}
\setbeamercolor*{block title example}{parent=example text,bg=black!15}
\setbeamertemplate{navigation symbols}{}
\setbeamercolor{footercolor}{fg=white,bg=black}

\makeatletter
\defbeamertemplate*{footline}{myfootline}
{
  \leavevmode%
  \hbox{%
  \begin{beamercolorbox}[wd=.333333\paperwidth,ht=2.25ex,dp=1ex,center]{footercolor}%
    \insertshorttitle
  \end{beamercolorbox}%
  \begin{beamercolorbox}[wd=.333333\paperwidth,ht=2.25ex,dp=1ex,center]{footercolor}%
    \insertshortauthor\expandafter\beamer@ifempty\expandafter{\beamer@shortinstitute}{}{~~(\insertshortinstitute)}
  \end{beamercolorbox}%
  \begin{beamercolorbox}[wd=.333333\paperwidth,ht=2.25ex,dp=1ex,right]{footercolor}%
    \insertshortdate{}\hspace*{2em}
    \insertframenumber{} / \inserttotalframenumber\hspace*{2ex} 
  \end{beamercolorbox}}%
  \vskip0pt%
}
\makeatother

\title{Girls PowerTech - Google Chrome Malware}

\institute{GoSecure}
\author{Lilly Chalupowski and Xandria Richman}
\date{May 3, 2018}

\begin{document}

\setbeamertemplate{footline}{}
\begin{frame}[t]
  \begin{center}
    \begingroup
    \fontsize{20pt}{20pt}\selectfont
    \inserttitle \\
    \endgroup
    \bigskip
      \includegraphics[scale=1]{title} \\
    \bigskip
    \insertauthor \\
    \insertdate
  \end{center}
\end{frame}

\newcommand\AtPagemyUpperLeft[1]{\AtPageLowerLeft{%
\put(\LenToUnit{0.94\paperwidth},\LenToUnit{0.93\paperheight}){#1}}}
\AddToShipoutPictureFG{
  \AtPagemyUpperLeft{{\includegraphics[width=1cm,keepaspectratio]{gosecure}}}
}%

\setbeamertemplate{footline}[myfootline]

\begin{frame}
  \frametitle{Introduction}
  \begin{center}
    \includegraphics[scale=0.4]{women_in_tech}
  \end{center}
\end{frame}

\begin{frame}
  \frametitle{Xandria Richman}
  \begin{columns}
    \begin{column}{0.5\textwidth}
      \includegraphics[scale=0.33]{xandria_richman}
    \end{column}
    \begin{column}{0.5\textwidth}
      \begin{center}
        \begin{tcolorbox}[title=Biography,colback=gray]
          \begin{itemize}
            {\color{black} \tiny
            \item Bachelor of Computer Science from UNB
            \item Volunteered for UNB to help introduce girls to technology
            \item Joined GoSecure in 2016 and has obtained a leadership position
            \item Found Flaws in Canada-wide loyalty card program
            \item Previously worked at Siemens Canada and IBM
            }
          \end{itemize}
        \end{tcolorbox}
      \end{center}
    \end{column}
  \end{columns}
\end{frame}

\begin{frame}
  \frametitle{Lilly Chalupowski}
  \begin{columns}
    \begin{column}{0.5\textwidth}
      \includegraphics[scale=0.5]{lilly_chalupowski}
    \end{column}
    \begin{column}{0.5\textwidth}
      \begin{center}
        \begin{tcolorbox}[title=Biography,colback=gray]
          \begin{itemize}
            {\color{black} \tiny
            \item Studied Computer Science and Audio Engineering at Acadia University
            \item Almost 100\% self taught on programming and hacking
            \item Presented at AtlSecCon on ROP Chain Exploitation
            \item Presented at AtlSecCon on Google Chrome Malware
            \item Taught SQL Injection and Phishing Awareness to Digital Nova Scotia Discovery Camp Kids
            \item Volunteers for Techsploration and mentoring girls looking to enter the tech industry
            \item Badger - ASLR Entropy and PE File Security Enumerator
            \item Chameleon - Custom Base64 Encoder
            \item Chrome Crusader - Google Chrome Malware and Botnet
            }
          \end{itemize}
        \end{tcolorbox}
      \end{center}
    \end{column}
  \end{columns}
\end{frame}

\begin{frame}
  \frametitle{What is Malware}
  \begin{center}
    \animategraphics[loop,controls,scale=0.4]{10}{img/malware/malware-}{0}{15}
    \begin{tcolorbox}[title=\href{https://en.wikipedia.org/wiki/Malware}{Definition of Malware},colback=gray]
      Software that is intended to damage or disable computers and computer systems.
    \end{tcolorbox}
  \end{center}
\end{frame}

\begin{frame}
  \frametitle{Google Chrome}
  \begin{columns}
    \begin{column}{0.5\textwidth}
      \includegraphics[scale=0.32]{google_chrome}
    \end{column}
    \begin{column}{0.5\textwidth}
      \begin{center}
        \begin{tcolorbox}[title=\href{https://en.wikipedia.org/wiki/Google_Chrome}{Google Chrome},colback=gray]
          {\small A freeware web browser developed by Google.}
        \end{tcolorbox}
      \end{center}
    \end{column}
  \end{columns}
\end{frame}

\begin{frame}
  \frametitle{Malware Inside Google Chrome}
  \framesubtitle{How is it possible?}
  \begin{center}
    \includegraphics[scale=0.2]{how}
  \end{center}
\end{frame}

\end{document}